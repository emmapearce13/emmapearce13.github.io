\chapter{Introduction}
\label{intro}


Here is an example of how to reference \citep[e.g.,][]{lange2016devito}.  And here is how we do it as part of a sentence as in \citet{lange2016devito}.


Here is how to insert a figure:

\begin{figure}[!htbp]
\begin{center}
\includegraphics*[angle=0, width=0.95\linewidth]{CHAPTERS/PICS/EAR.pdf}
\end{center}
\caption[Here is a caption that goes in the list of figures]
{This part of the caption goes below the figure itself.}
\label{EAR}
\end{figure}

And now that it has a label, you can reference it as Figure \ref{EAR}.


If you want to crop a pdf image to fit into a LaTeX document, you could use  pdfcrop to do it.

\verb|pdfcrop file.pdf|

and note that if you want to create a pdf file from ps files that emerge from GMT, you can do it like this:

\verb|ps2pdf file.ps|

\verb|pdfcrop file.pdf|

Adding a couple of lines like this in your GMT scripts would create a LaTeX ready file.

And here is how to make a table:

\begin{table}
\caption[Here is a caption that goes in the list of Tables]
{This part of the caption goes below the Table itself.}
\begin{center}
\begin{tabular}{lllrll}
\hline
Station & Lat. & Lon. & Elev. (m) & Network & Location \\
\hline
CRLN & 64.189 & -83.348 & 66  & HuBLE-UK & Southampton Island 	\\
CTSN & 62.852 & -82.485 & 41  & HuBLE-UK & Coats Island 		\\
\hline
\end{tabular}
\end{center}
\label{some_data}
\end{table}
\newpage


Here are some equations using maths mode (you may remember these from seismology in second year):

\begin{equation}
t_{1}= H\left[{\sqrt{\frac{1}{V_{S}^{2}}-p^{2}} - \sqrt{\frac{1}{V_{P}^{2}}-p^{2}}} \,\right]
\label{RFStravelT1}
\end{equation}

\begin{equation}
t_{2}=H\left[{\sqrt{\frac{1}{V_{S}^{2}}-p^{2}} + \sqrt{\frac{1}{V_{P}^{2}}-p^{2}}}\,\right]
\label{RFStravelT2}
\end{equation}

\begin{equation}
t_{3}=2H\sqrt{\frac{1}{V_{S}^{2}}-p^{2}}
\label{RFStravelT3}
\end{equation}

where $p$ is the ray parameter (skm$^{-1}$).  Note that maths mode can also be done within the text, which can be quite useful when referring to splitting parameters such as $\phi$ and $\delta$t.






