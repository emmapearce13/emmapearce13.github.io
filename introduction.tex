\chapter{Introduction}
\label{intro}


Full Waveform Inversion (FWI) is a numerical method that provides high resolution quantitative images obtained via seismic exploration. It is a mix of the most successful parts of seismic reflection and seismic refraction, providing data that can specify information of the geometrical distribution of geological features along with quantitative images of the Earth's physical properties. 
It uses the two-way wave equation to predict seismic data from a model and aims to find a model that matches the raw data wiggle-for-wiggle by reducing the difference between the predictions and the observed field data (pre-stacked) \citep{tarantola2005inverse}. \\
 
Historically, seismic techniques have been used as a method of imaging the interior of the Earth's surface, with the first attempt recorded as early as the end of the $19^{th}$ century. From here full-waveform inversion has evolved due to the contribution of multiple scientists and is now the preferred seismic method amongst geophysicists due to its high degree of spatial resolution and cost effectiveness when imaging large 3D volumes. FWI is therefore being expanded to dimensions that better represent real life examples. In 2011 \citet{guasch20123d} completed his PhD thesis expanding FWI to three dimensions with Elastic properties being taken into account, programming extensive numerical solutions to the seismic wave equation in Fortran. Although successful, to implicate FWI in Fortran requires a detailed knowledge of the program along with many lines of code. Due to this FWI has been re-coded utilising the work of \citep{guasch20123d} but implemented in Python by Opesci  \citep{bworld}.  This enables equations to be written using Sybolic Python requiring only a few lines of code to achieve the same result. \\

This project will implement and test the work by Opesci for FWI and compare the results to those obtained by \citep{guasch20123d} for the same data to establish variations between the two methods. The examples of FWI in this report apply to 2D small-amplitude pressure waves propagating within an inhomogeneous, isotropic, non-attenuating, non-dispersive, stationary, fluid medium. Although relatively easy to add 3D, anisotropy, attenuation and elastic effects as the general approach stays the same, the numerical complexity increases which in turn leads to larger computational power and memory being required, which with the allocated time scale of this project was not possible. 
 

 